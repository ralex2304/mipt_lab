\documentclass[a4paper, 12pt]{article}
\usepackage[a4paper,top=1.5cm, bottom=1.5cm, left=1cm, right=1cm]{geometry}
\usepackage{cmap}					% поиск в PDF
\usepackage{mathtext} 				% русские буквы в формулах
\usepackage[T2A]{fontenc}			% кодировка
\usepackage[utf8]{inputenc}			% кодировка исходного текста
\usepackage[english,russian]{babel}	% локализация и переносы

\usepackage{amsmath,amssymb}
\usepackage{indentfirst}
\usepackage{longtable}
\usepackage{graphicx}
\usepackage{array}
\usepackage{float}

\usepackage{floatflt}
\usepackage{wrapfig}
\usepackage{siunitx} % Required for alignment
\usepackage{subfig}
\usepackage{multirow}
\usepackage{rotating}
\usepackage{caption}

\graphicspath{{.}}


\title{\begin{center}Лабораторная работа №5.2.2(2.3)\end{center}
Изучение спектров атомарного водорода и йода}
\author{Рожков А. В.}
\date{\today}

\begin{document}
    \pagenumbering{gobble}
    \maketitle
    \newpage
    \pagenumbering{arabic}

    \textbf{Цель работы:} исследовать спектральные закономерности в оптических спектрах водорода; по результатам измерений вычислить постоянные Ридберга для водорода; исследовать спектр поглощения паров йода в видимой области; по результатам измерения вычислить энергию колебательного кванта молекулы йода и энергию её диссоциации в основном и возбужденном состояниях.

    \textbf{В работе используются:} стеклянно-призменный монохроматор УМ-2, неоновая лампа, ртутная лампа ПРК-4, водородная лампа, кювета с кристаллами йода с лампой накаливания.

    \section{Теоретические сведения}
        \subsection*{Изучение спектра водорода}
            Атом водорода является простейшей квантовой системой, для которой уравнение Шрёдингера может быть решено точно. Это также верно для водородноподобных атомов, то есть атомов с одним электроном на внешней оболочке.
            Если считать ядро неподвижным, то эти энергетические уровни определяются выражением
            \begin{equation}
                E_n = - \frac{m_e (Z e^2)^2}{2\hbar^2}\frac{1}{n^2},
            \end{equation}
            где $n$ есть номер энергетического уровня, $Z$ есть зарядовое число ядра рассматриваемого атома, которое в случае атома водорода равно 1.\\
            При переходе электрона с $m$-го на $n$-й уровень излучается фотон с энергией
            \begin{equation}
                E_\gamma = E_m - E_n = \frac{m_ee^2}{2\hbar^2}Z^2\left(\frac{1}{n^2} - \frac{1}{m^2}\right) = Ry Z^2 (\frac{1}{n^2} - \frac{1}{m^2}).
            \end{equation}
            Длина волны  соответствующего излучения $\lambda_{mn}$ связана с номерами уровней следующим соотношением:
            \begin{equation}
                \label{eq:Ry}
                \frac{1}{\lambda_{mn}} =\frac{m_ee^2}{4\pi\hbar^3c}Z^2\left(\frac{1}{n^2}-\frac{1}{m^2}\right) = \text{R} Z^2 \left(\frac{1}{n^2}-\frac{1}{m^2}\right),
            \end{equation}
            где $\text{R} = \cfrac{m_ee^2}{4\pi\hbar^3c}$ есть постоянная Ридберга.

            В данной работе будет изучаться серия Бальмера атома водорода, в которой электроны совершают переходы с некоторого уровня $m$ на уровень $n = 2$.
        \subsection*{Спектр поглощения йода}
            В первом приближении энергия молекулы может быть представлена в виде:
            \begin{equation}
                E = E_{\text{эл}}+E_{\text{колеб}}+E_{\text{вращ}},
            \end{equation}
            где $E_{\text{эл}}$ есть энергия электронных уровней, $E_{\text{колеб}}$ есть энергия колебательных уровней, $E_{\text{вращ}}$ есть энергия вращательных уровней.

            В настоящей работе рассматриваются оптические переходы, то есть переходы, связанные с излучением фотонов в видимом диапазоне длин волн. Они соответствуют переходам между различными электронными состояниями. При этом также происходят изменения вращательного и колебательного состояний, однако в реальности ввиду малости характерных энергий вращательные переходы ненаблюдаемы.

            Более конкретно, изучаются переходы из колебательного состояния с номером $n_1$ основного электронного уровня с энергией $E_1$ в колебательное состояние с номером $n_2$ на электронный уровень с энергией $E_2$. Энергия таких переходов описывается формулой:
            \begin{equation}
                h \nu_{n_1,n_2}=(E_2-E_1)+h\nu_2(n_2+\dfrac{1}{2})-h \nu_1(n_1+\dfrac{1}{2}),
            \end{equation}
            где $h\nu_1$ и $h\nu_2$ суть энергии колебательных квантов на электронных уровнях с энергиями $E_1$ и $E_2$.

            При достаточно больших квантовых числах $n_1$ и $n_2$ колебательные уровни переходят в непрерывный спектр, что соответствует диссоциации молекулы. Наименьшая энергия, которую нужно сообщить молекуле в нижайшем колебательном состоянии, чтобы она диссоциировала, называется энергией диссоциации.

            В данной работе определяются энергии диссоциации на первых двух электронных уровнях.

            \begin{figure}[h!]
                \begin{center}
                \includegraphics[width = 0.5\textwidth]{img/potential_curves.png}
                \caption{Потенциальные кривые и характерные электронно-колебательные переходы}
                \label{potent_curves}
                \end{center}
            \end{figure}

            \subsubsection*{Серии Деландра в спектре йода}

                В данной работе изучается электронно-колебательный спектр поглощения паров йода I$_2$ в видимой области при температуре $T \approx 300\ \text{K}$. Основной вклад дают переходы  между колебательными подуровнями двух соседних электронных состояний:
                \[
                (1, n_1) \rightarrow (2, n_2),
                \]
                где индекс «1» обозначает основное электронное состояние, «2» — возбуждённое.

                Все возможные линии поглощения удобно разбить на \textit{серии Деландра}, каждая из которых соответствует фиксированному начальному колебательному уровню (например, $n_1 = 0$ или $n_1 = 1$), а конечный уровень $n_2 = 0, 1, 2, \ldots$ меняется:

                \begin{itemize}
                    \item 0-я серия: переходы из $n_1 = 0$ в $n_2 = 0, 1, 2, \ldots$;
                    \item 1-я серия: переходы из $n_1 = 1$ в $n_2 = 0, 1, 2, \ldots$.
                \end{itemize}

                При температуре около комнатной относительные заселённости колебательных уровней подчиняются распределению Больцмана:
                \[
                N_n \propto e^{-E_n/kT}.
                \]

                Расчёт показывает, что при $T \approx 300\ \text{K}$ выполняется примерно
                \[
                N_0 : N_1 : N_2 \approx 1 : \frac{1}{3} : \frac{1}{10},
                \]
                поэтому наибольший вклад дают 0-я и 1-я серии Деландра.

            \begin{figure}[h!]
                \begin{center}
                \includegraphics[width = 0.5\textwidth]{img/delandr.png}
                \caption{Спектр поглощения паров йода}
                \label{delandr}
                \end{center}
            \end{figure}

    \newpage

    \section{Экспериментальная установка}
        \subsection{Измерение серии Бальмера}

            \begin{figure}[h!]
                \begin{center}
                \includegraphics[width = 0.5\textwidth]{img/setup_1.png}
                \caption{Устройство монохроматора УМ-2}
                \label{setup1}
                \end{center}
            \end{figure}

            Для измерения длин волн спектральных линий используется монохроматор УМ-2, предназначенный для спектральных исследований в диапазоне от $0,38$ до $1,00$ мкм. Его устройство приведено на рис. \ref{setup1}.

            В качестве источника используется водородная лампа в виде H-образной трубки, питаемая от катушки Румкорфа

        \subsection{Измерение спектр поглощения паров йода}

            \begin{figure}[h!]
                \begin{center}
                \includegraphics[width = 0.5\textwidth]{img/setup_2.png}
                \caption{Схема экспериментальной установки для получения спектра поглощения}
                \label{setup2}
                \end{center}
            \end{figure}

            Для получения спектра поглощения необходимы: 1) источник сплошного спектра - лампа накаливания;
            2) поглощающая среда - кювета с исследуемым веществом;
            3) спектральный прибор, регистрирующий спектр поглощения - монохроматор УМ-2.

    \section{Ход работы}

        \subsection{Калибровка спектрометра}

            Откалибруем спектрометр УМ-3. Для этого установим неоновую лампу и отцентрируем и отъюстируем систему, таким образом,
            что изображение щели (спектр) и стрелка находилась в одной плоскости, при этом стрелка и спектр должны смещаться вместе при смещении точки наблюдения по горизонтали.

            Рассмотрим 25 спектральных линий неона и поставим им в соответствие значение на барабане. Точность измерений угла поворота барабана $\sigma_d = 1^\circ$.

            \begin{table}[!ht]
                \centering
                \begin{tabular}{|c|c|}
                    \hline

                    $\theta$ & $\lambda, $\AA\\ \hline
                    $2918 \pm 2$ & 7032\\ \hline
                    $2892 \pm 2$ & 6929\\ \hline
                    $2824 \pm 2$ & 6717\\ \hline
                    $2812 \pm 2$ & 6678\\ \hline
                    $2786 \pm 2$ & 6599\\ \hline
                    $2764 \pm 2$ & 6533\\ \hline
                    $2754 \pm 2$ & 6507\\ \hline
                    $2712 \pm 2$ & 6402\\ \hline
                    $2704 \pm 2$ & 6383\\ \hline
                    $2688 \pm 2$ & 6334\\ \hline
                    $2676 \pm 2$ & 6305\\ \hline
                    $2662 \pm 2$ & 6267\\ \hline
                    $2642 \pm 2$ & 6217\\ \hline
                    $2620 \pm 2$ & 6164\\ \hline
                    $2610 \pm 2$ & 6143\\ \hline
                    $2592 \pm 2$ & 6096\\ \hline
                    $2582 \pm 2$ & 6074\\ \hline
                    $2562 \pm 2$ & 6030\\ \hline
                    $2538 \pm 2$ & 5976\\ \hline
                    $2524 \pm 2$ & 5945\\ \hline
                    $2492 \pm 2$ & 5882\\ \hline
                    $2478 \pm 2$ & 5852\\ \hline
                    $2214 \pm 2$ & 5401\\ \hline

                \end{tabular}
                \caption{Спектр неона}
                \label{Neon}
            \end{table}

            Аналогично поступаем для ртутной лампы.

            \begin{table}[!ht]
                \centering
                \begin{tabular}{|c|c|}
                    \hline

                    $\theta$ & $\lambda, $\AA\\ \hline
                    $2882 \pm 2$ & 6907\\ \hline
                    $2648 \pm 2$ & 6234\\ \hline
                    $2446 \pm 2$ & 5791\\ \hline
                    $2436 \pm 2$ & 5770\\ \hline
                    $2256 \pm 2$ & 5461\\ \hline
                    $1834 \pm 2$ & 4916\\ \hline
                    $1216 \pm 2$ & 4358\\ \hline
                    $614 \pm 2$ & 4047\\ \hline

                \end{tabular}
                \caption{Спектр ртути}
                \label{Hg}
            \end{table}

            Построим градуировочный график по полученным значениям спектральных линий неона и ртути. Искать зависимость $\lambda = f(\theta)$ будем в виде (дисперсионная формула Гартмана):
            $\lambda = a + \frac{b}{\theta - c}$

            \begin{figure}[h!]
                \begin{center}
                \includegraphics[width = 0.5\textwidth]{img/calibration.png}
                \caption{Градуировочный график спектрометра УМ-2}
                \end{center}
            \end{figure}

            Коэффициенты аппроксимации:

            $$
                a = (2310 \pm 20)~\text{\AA}
            $$
            $$
                b = (-627 \pm 7) \cdot 10^4~\text{\AA}
            $$
            $$
                c = (4248 \pm 11)
            $$

            Погрешность по формуле $\sigma_{\lambda} = \sigma_{\theta} \frac{d\lambda}{d\theta}$

        \subsection{Спектральные линии водорода}

            Измерим положение линий водорода $H_{\alpha}$, $H_{\beta}$, $H_{\gamma}$, $H_{\delta}$ и с помощью градуировочного графика определим длины волн.

            \begin{table}[!ht]
                \centering
                \begin{tabular}{|c|c|c|c|c|}
                    \hline

                     & $H_{\alpha}$ & $H_{\beta}$ & $H_{\gamma}$ & $H_{\delta}$\\ \hline
                    $\theta$ & $2772 \pm 2$ & $1780 \pm 2$ & $1138 \pm 2$ & $710 \pm 2$\\ \hline
                    $\lambda, $\AA & $6561 \pm 6$ & $4852 \pm 2$ & $4327 \pm 1$ & $4082 \pm 1$\\ \hline
                    $\lambda_0, $\AA & $6563$ & $4861$ & $4341$ & $4102$\\ \hline
                    $R, см^{-1}$ & $109740 \pm 10$ & $109929 \pm 9$ & $110061 \pm 7$ & $110227 \pm 6$\\ \hline

                \end{tabular}
                \caption{Результаты измерения для водорода}
                \label{}
            \end{table}

            Полученные значения для спектральных линий водорода серии Бальмера совпадают с табличными значениями $\lambda_0$.

            Для серии Бальмера $n = 2$ величина $m$ для первых четырех линий этой серии принимает значения $3$, $4$, $5$, $6$.

            Для водорода
            $$
                R = \frac{1}{\lambda_{mn}} \cdot \frac{m^2 n^2}{m^2 - n^2}
            $$

            Среднее значение постоянной Ридберга для водорода $R_{ср} = (109990 \pm 100)~см^{-1}$.

            Табличное значение постоянной Ридберга для водорода $R = 109677~см^{-1}$.

        \subsection{Спектр поглощения йода}

            Теперь рассмотрим спектр поглощения йода. Измерим положение самой длинноволновой из наблюдаемых линий поглощения $h\nu_{1,0}$,
            шестой по счёту от длинноволнового края $h\nu_{1,5}$ и границу схождения спектра $h\nu_{гр}$. Из градуировочного графика определяем соответствующие длины волн.

            $$
                \lambda_{1,0} = (6236 \pm 5)~\text{\AA}
            $$
            $$
                \lambda_{1,5} = (6036 \pm 4)~\text{\AA}
            $$
            $$
                \lambda_{гр} = (5133 \pm 3)~\text{\AA}
            $$

            Определим энергию колебательного кванта возбужденного состояния молекулы йода:
            $$
                h\nu_2 = \frac{hv_{1,5} - hv_{1,0}}{5} = (0.0132 \pm 0.0004) \text{ эВ}
            $$

            Энергия колебательного кванта основного состояния равно $h\nu_1 = 0.027 \text{ эВ}$, энергия возбуждения атома равна $E_a = 0.94 \text{ эВ}$. Вычислим энергию электронного перехода $h\nu_{\text{эл}}$, энергию диссоциации молекулы в основном состоянии $D_1$ и в возбужденном состоянии $D_2$.

            \[ h \nu_{1,0} =  h \nu_{\text{эл}} + \frac{1}{2} h \nu_2 - \frac{3}{2} h \nu_1 \Rightarrow h \nu_{\text{эл}} = (2.022 \pm 0.002) \text{ эВ}\]
            \[ D_1 + E_a = h \nu_{\text{гр}} \Rightarrow D_1 = (1.4755 \pm 0.0012) \text{ эВ} \]
            \[ D_2 + h \nu_{\text{эл}} = h \nu_{\text{гр}} \Rightarrow D_2 = (0.393 \pm 0.002) \text{ эВ}\]

            Полученные результаты совпадают с табличными в пределах погрешности.

    \section{Вывод}
        В данной работе исследовались спектральные закономерности в оптическом спектре водорода и спектре поглощения йода в видимой области.

        С помощью табличных данных о спектрах неона и ртути был проградуирован монохроматор УМ-2.

        Получены длины волн серии Бальмера для водорода и вычислена постоянная Ридберга.

        Вычислены энергия колебательного кванта возбужденного состоянии молекулы йода, энергии электронного перехода, энергии диссоциации молекулы в основном и возбужденном состояниях.

\end{document}

