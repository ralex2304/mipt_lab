\documentclass[a4paper, 12pt]{article}
\usepackage[a4paper,top=1.5cm, bottom=1.5cm, left=1cm, right=1cm]{geometry}
\usepackage{cmap}					% поиск в PDF
\usepackage{mathtext} 				% русские буквы в формулах
\usepackage[T2A]{fontenc}			% кодировка
\usepackage[utf8]{inputenc}			% кодировка исходного текста
\usepackage[english,russian]{babel}	% локализация и переносы

\usepackage{amsmath,amssymb}
\usepackage{indentfirst}
\usepackage{longtable}
\usepackage{graphicx}
\usepackage{array}
\usepackage{float}

\usepackage{floatflt}
\usepackage{wrapfig}
\usepackage{siunitx} % Required for alignment
\usepackage{subfig}
\usepackage{multirow}
\usepackage{rotating}
\usepackage{caption}

\graphicspath{{.}}


\title{\begin{center}Лабораторная работа №5.1.1\end{center}
Экспериментальная проверка уравнения Эйнштейна для фотоэффекта и определение постоянной Планка}
\author{Рожков А. В.}
\date{\today}

\begin{document}
    \pagenumbering{gobble}
    \maketitle
    \newpage
    \pagenumbering{arabic}

    \textbf{Цель работы:} исследовать зависимость фототока от величины задерживающего потенциала и частоты падающего излучения, что позволяет вычислить величину постоянной Планка.

    \textbf{В работе используются:} источник света, конденсор, монохроматор УМ-2, фотоэлемент с усилителем постоянного тока.

    \section{Теоретические сведения}

        Фотоэффект -- явление испускания электронов фотокатодом, облучаемым светом. Взаимодействие монохроматического света с веществом можно описывать как взаимодействие с веществом частиц, называемых фотонами, которые обладают энергией $\hbar \omega$ и импульсом $\hbar \omega /c$. При столкновении фотона с электроном фотокатода энергия фотона полностью передается электрону, и фотон прекращает свое существование. Энергетический баланс этого взаимодействия для вылетающих электронов описывается уравнением

        \begin{equation}
            \hbar \omega = E_{max} + W
            \label{energy balance}
        \end{equation}

        \begin{figure}[!ht]
            \centering
            \includegraphics[width = 0.35\textwidth]{img/I_V_th.png}
            \caption{Зависимость фототока от напряжения на аноде фотоэлемента}
            \label{pict I(V)}
        \end{figure}

        Здесь $E_{max}$ -- максимальная кинетическая энергия электрона после выхода из фотокатода, $W$ -- работа выхода электрона из катода. Реально энергетический спектр вылетевших из фотокатода электронов непрерывен -- он простирается от нуля до $E_{max}$.

        Для измерения энергии вылетевших фотоэлектронов вблизи фотокатода обычно располагается второй электрод (анод), на который подается задерживающий ($V < 0$) или ускоряющий ($V > 0$) потенциал. При достаточно больших ускоряющих напряжениях фототок достигает насыщения (рис. \ref{pict I(V)}): все испущенные электроны попадают на анод.

        При задерживающих потенциалах на анод попадают лишь электроны, обладающие достаточно большой кинетической энергией, в то время как медленно движущиеся электроны заворачиваются полем и возвращаются на катод. При некотором значении $V = -V_0$ (потенциал запирания) даже наиболее быстрые фотоэлектроны не могут достичь анода.

        Максимальная кинетическая энергия $ E_{max} $ электронов связана с запирающим потенциалом $V_0$ очевидным соотношением $E_{max} = eV_0$. Тогда \eqref{energy balance} примет вид, называемый уравнением Эйнштейна:

        \begin{equation}
            eV_0 = \hbar\omega - W
            \label{Einstein}
        \end{equation}

        Чтобы определить величину запирающего напряжения, нам надо правильно экстраполировать получаемую токовую зависимость к нулю, т. е. определить, какова функциональная зависимость $I(V)$. Расчет для простейшей геометрии -- плоский катод, освещаемый светом, и параллельный ему анод -- приводит к зависимости

        \begin{equation}
            \sqrt{I} \propto V_0 - V,
            \label{sqrt I = V}
        \end{equation}
        т. е. корень квадратный из фототока линейно зависит от запирающего напряжения. Эта зависимость хорошо описывает экспериментальные данные.

        В работе изучается зависимость фототока из фотоэлемента от величины задерживающего потенциала $V$ для различных частот света $\omega$, лежащих в видимой области спектра.

        \begin{figure}[!ht]
            \centering
            \includegraphics[width = 0.3\textwidth]{img/V_omega_th.png}
            \caption{Зависимость запирающего потенциала от частоты света}
            \label{pict V(w)}
        \end{figure}

        Потенциал запирания $V_0$ для любого катода линейно зависит от частоты света $\omega$:

        \begin{equation}
            V_0 (\omega) = \frac{\hbar \omega - W}{e}
            \label{V_0_omega}
        \end{equation}

        По наклону прямой на графике $V_0(\omega)$ (рис. \ref{pict V(w)}) можно определить постоянную Планка:

        \begin{equation}
            \dfrac{dV_0}{d\omega} = \dfrac{\hbar}{e}
            \label{dVdw}
        \end{equation}

        Как показывает формула \eqref{dVdw}, угол наклона прямой $V_0(\omega) $ не зависит от рода вещества, из которого изготовлен фотокатод. От рода вещества, однако, зависит величина фототока, работа выхода $W$ и форма кривой $I(V)$ (рис. \ref{pict I(V)}). Все это определяет выбор пригодных для опыта катодов.

    \section{Экспериментальная установка}
        \begin{figure}[!ht]
            \centering
            \includegraphics[width = 0.6\textwidth]{img/setup.png}
            \caption{Принципиальная схема экспериментальной установки}
            \label{exp_scheme}
        \end{figure}

        Свет от источника $S$ (обычная электрическая лампа накаливания) с помощью конденсора фокусируется на входную щель призменного монохроматора УМ-2, выделяющего узкий спектральный интервал, и попадает на катод фотоэлемента ФЭ.

        Фотоэлемент конструктивно представляет собой откачанный до высокого вакуума стеклянный баллон диаметром 25 мм и высотой 30 мм. Внутри баллона расположены два электрода: фотокатод и анод. Фотокатод представляет собой тонкую пленку металла, легированного элементами $Na$, $K$, $Sb$ и $Cs$ и расположенного на массивной металлической пластине. Анод фотоэлемента выполнен в виде пояска тонкой пленки, осажденной на внутренней части боковой поверхности вверху баллона. Такое расположение фотокатода и анода обеспечивает наиболее полный сбор на аноде электронов, эмитированных фотокатодом. Такой фотоэлемент обладает спектральной чувствительностью в области длин волн от 300 до 850 нм.

        Фототок, протекающий в фотоэлементе, мал, особенно при потенциалах $V$, близких к $V_0$ , и не может быть измерен непосредственно. Для его измерения используется усилитель постоянного тока.  Абсолютные значения фототока нам не нужны, поэтому он измеряется в относительных единицах цифровым вольтметром $V_2$ , подключенным к выходу усилителя. Эти показания пропорциональны величине измеряемого тока. Измерение тормозящего потенциала осуществляется с помощью цифрового вольтметра $V_1$.

        Контактная разность потенциалов между катодом и анодом мешает точному определению величины $V_0$, но не оказывает влияния на определение постоянной Планка, которая выражается через производную $dV_0 / d\omega$.

        \subsetion{Призма постоянного отклонения (призма Аббе)}

            \begin{figure}[!ht]
                \centering
                \includegraphics[width = 0.5\textwidth]{img/abbe.png}
                \caption{Ход лучей в призме Аббе}
                \label{exp_scheme}
            \end{figure}

            В монохроматоре использована призма Аббе. Её можно разделить на 3 призмы: 2 прямоугольные полупризмы с преломляющим углом $30^o$, третья призма полного внутреннего отражения, отклоняющая лучи на $90^o$. Дисперсия происходит лишь на гранях полупризм.

            Если на призму падает немонохроматический пучок параллельных лучей, то, вращая призму вокруг оси, перпендикулярной к плоскости чертежа (этим мы меняем угол падения), всегда можно добиться того, что монохроматический свет любой длины волны будет выходить из призмы под углом $90^o$ к падающему лучу.

    \section{Ход работы}

        \subsection{Градуировка монохроматора}

            Проведём градуировку монохроматора по неоновой лампе с известным спектром. Результаты в таблице \ref{cal_tab} и на рис. \ref{cal_plot}. Аппроксимацию проводим полиномом третьей степени:

            $$
                \lambda = 9.56 * 10^{-7}~\theta^3 - 5.99 * 10^{-3}~\theta^2 + 14.0~\theta - 6.62 * 10^3
            $$

            \begin{figure}[!ht]
                \centering
                \includegraphics[width = 0.6\textwidth]{img/calibration.png}
                \caption{Градуировочный график монохроматора}
                \label{cal_plot}
            \end{figure}

            \begin{table}[!ht]
                \centering
                \begin{tabular}{|c|c||c|c|}
                    \hline

                    $\theta$ & $\lambda,$ \AA & $\theta$ & $\lambda,$ \AA \\ \hline
                    $2918 \pm 2$ & $7032$ & $2618 \pm 2$ & $6164$ \\ \hline
                    $2892 \pm 2$ & $6929$ & $2608 \pm 2$ & $6143$ \\ \hline
                    $2824 \pm 2$ & $6717$ & $2588 \pm 2$ & $6096$ \\ \hline
                    $2812 \pm 2$ & $6678$ & $2578 \pm 2$ & $6074$ \\ \hline
                    $2786 \pm 2$ & $6599$ & $2558 \pm 2$ & $6030$ \\ \hline
                    $2762 \pm 2$ & $6533$ & $2532 \pm 2$ & $5976$ \\ \hline
                    $2754 \pm 2$ & $6507$ & $2518 \pm 2$ & $5945$ \\ \hline
                    $2714 \pm 2$ & $6402$ & $2488 \pm 2$ & $5882$ \\ \hline
                    $2706 \pm 2$ & $6383$ & $2472 \pm 2$ & $5852$ \\ \hline
                    $2688 \pm 2$ & $6334$ & $2210 \pm 2$ & $5401$ \\ \hline
                    $2676 \pm 2$ & $6305$ & $2168 \pm 2$ & $5341$ \\ \hline
                    $2662 \pm 2$ & $6267$ & $2162 \pm 2$ & $5331$ \\ \hline
                    $2638 \pm 2$ & $6217$ \\ \cline{1-2}

                \end{tabular}
                \caption{Градуировочная таблица монохроматора}
                \label{cal_tab}
            \end{table}

            \newpage

        \subsection{Исследование зависимости тока фототока от напряжения на аноде}

            Для определения диапазона дальнейших измерений грубо исследуем зависимость по всей области при $\lambda = 6239$ \AA. В работе измерялся не фототок напрямую, а пропорциональное ему напряжение $U_i$. Результат на рис. \ref{plot:rough}. Оптимальным для более точных измерений является участок до $U \approx 1~В$.

            \begin{figure}[!ht]
                \centering
                \includegraphics[width = 0.6\textwidth]{img/full.png}
                \caption{График зависимости фототока от напряжения на аноде}
                \label{plot:rough}
            \end{figure}

        \subsection{Определение потенциала запирания для разных длин волн путём экстраполяции}

            Построим графики зависимости $\sqrt{U_i} = f(U)$, которые аппроксимируем прямыми и экстраполируем до пересечения с осью абсцисс. Таким образом получим потенциалы запирания. Полные результаты на графиках и в таблицах в приложении. Кратко результаты представлены на рис. \ref{plot:sqrt_all} и в таблице \ref{tab:U_0_res}.

            Погрешность измерения $U$ равна $0.0001~В$, $U_i$ -- $1~мВ$.

            При аппроксимации методом наименьших квадратов были использованы не все измеренные точки, а только 5-6 наиболее близких к 0, так как далее линейность зависимости сильно нарушается

            \begin{figure}[!ht]
                \centering
                \includegraphics[width = 0.8\textwidth]{img/sqrt_plot.png}
                \caption{Графики зависимости корня фототока от анодного напряжения, аппроксимация}
                \label{plot:sqrt_all}
            \end{figure}

            \begin{table}[!ht]
                \centering
                \begin{tabular}{|c|c|c|c|c|c|}
                    \hline

                    $\theta$ & $\lambda,$ \AA & $\omega, 10^{15} рад/с$ & $k, 10^{-3} В^{-1/2}$ & $b, В^{1/2}$ & $U_0, В$\\ \hline
                    $2210 \pm 2$ & $5404 \pm 11$ & $3.488 \pm 0.007$ & $8.8 \pm 1.0$ & $4.9 \pm 0.3$ & $0.56 \pm 0.07$\\ \hline
                    $2310 \pm 2$ & $5564 \pm 11$ & $3.388 \pm 0.006$ & $7.0 \pm 0.5$ & $4.64 \pm 0.11$ & $0.66 \pm 0.05$\\ \hline
                    $2410 \pm 2$ & $5737 \pm 10$ & $3.285 \pm 0.006$ & $10 \pm 1$ & $4.5 \pm 0.3$ & $0.46 \pm 0.06$\\ \hline
                    $2510 \pm 2$ & $5929 \pm 10$ & $3.179 \pm 0.005$ & $9.9 \pm 0.7$ & $4.3 \pm 0.2$ & $0.43 \pm 0.04$\\ \hline
                    $2610 \pm 2$ & $6145 \pm 10$ & $3.067 \pm 0.005$ & $10 \pm 1$ & $3.5 \pm 0.2$ & $0.36 \pm 0.04$\\ \hline
                    $2710 \pm 2$ & $6391 \pm 10$ & $2.949 \pm 0.004$ & $10 \pm 1$ & $2.6 \pm 0.2$ & $0.26 \pm 0.04$\\ \hline
                    $2810 \pm 2$ & $6673 \pm 9$ & $2.825 \pm 0.004$ & $11 \pm 2$ & $1.7 \pm 0.3$ & $0.16 \pm 0.04$\\ \hline
                    $2918 \pm 2$ & $7025 \pm 9$ & $2.683 \pm 0.004$ & $9 \pm 2$ & $0.3 \pm 0.4$ & $0.03 \pm 0.04$\\ \hline

                \end{tabular}
                \caption{Таблица результатов запирающих потенциалов}
                \label{tab:U_0_res}
            \end{table}

            При помощи следующей формулы находим $U_0$:

            $$
                U_0 = \frac{-b}{k}
            $$

        \subsection{Определение постоянной Дирака по полученным данным}

        При помощи графика $U_0 = f(\omega)$ на рис. \ref{plot:U_omega} ($y = k*x + b$) можно определить постоянную Дирака. По МНК $k = (7.2 \pm 0.8)*10^{-16}~В*c$; $b = (-1.9 \pm 0.2)~В$.

            \begin{figure}[!ht]
                \centering
                \includegraphics[width = 0.8\textwidth]{img/U_omega.png}
                \caption{График зависимости запирающего напряжения от угловой частоты волны падающего света}
                \label{plot:U_omega}
            \end{figure}

            $$
                \hbar = \frac{dU_0}{d\omega} * e = k * e = (7.2 \pm 0.8) * 10^{-16}~эВ*с
            $$

            Точка пересечения графика с осью абсцисс отражает красную границу фотоэффекта. Из формулы (\ref{V_0_omega}) можем определить работу выхода материала. Однако данные величины могут быть найдены только с точностью до контактной разницы потенциалов.

            $$
                W = -b * e = (1.9 \pm 0.2)~эВ
            $$
            $$
                \omega_{кр} = \frac{-b}{k} = (2.6 \pm 0.4) * 10^{15}~рад/с
            $$

    \section{Вывод}

        Экспериментально получили постоянную Дирака $\hbar = (7.2 \pm 0.8) * 10^{-16}~эВ*с$. Результат совпал с табличным ($6.582 * 10^{-16}~эВ*с$) с точностью до погрешности.

        Полученное значение красной границы $\omega_{кр} = (2.6 \pm 0.8) * 10^{15} рад/с$ почти совпало с табличным для $Na_2 K Sb (Cs)$ ($2.09 * 10^{15} рад/c$). Различие объясняется тем, что данная величина была получена с точностью до контактной разницы потенциалов.

        Получили значение работы выхода $W = (1.9 \pm 0.2) эВ$.

    \newpage

    \section{Приложение}

        \begin{figure}[ht!]
            \centering
            \subfloat[\centering]{{\includegraphics[width=0.47\linewidth]{img/sqrt_plot_2210.png}}}
            \qquad
            \subfloat[\centering]{{\includegraphics[width=0.47\linewidth]{img/sqrt_plot_2310.png}}}
        \end{figure}

        \begin{figure}[ht!]
            \centering
            \subfloat[\centering]{{\includegraphics[width=0.47\linewidth]{img/sqrt_plot_2410.png}}}
            \qquad
            \subfloat[\centering]{{\includegraphics[width=0.47\linewidth]{img/sqrt_plot_2510.png}}}
        \end{figure}

        \begin{figure}[ht!]
            \centering
            \subfloat[\centering]{{\includegraphics[width=0.47\linewidth]{img/sqrt_plot_2610.png}}}
            \qquad
            \subfloat[\centering]{{\includegraphics[width=0.47\linewidth]{img/sqrt_plot_2710.png}}}
        \end{figure}

        \begin{figure}[ht!]
            \centering
            \subfloat[\centering]{{\includegraphics[width=0.47\linewidth]{img/sqrt_plot_2810.png}}}
            \qquad
            \subfloat[\centering]{{\includegraphics[width=0.47\linewidth]{img/sqrt_plot_2918.png}}}
        \end{figure}

        \begin{minipage}{.23\linewidth}
            \centering
            \begin{tabular}{|c|c|c|}
                \hline

                $U, В$ & $U_i, мВ$ & $\sqrt{U_i}, мВ^{1/2}$\\ \hline
                $1.0000$ & $92$ & $9.59 \pm 0.05$\\ \hline
                $0.9000$ & $81$ & $9.00 \pm 0.06$\\ \hline
                $0.8000$ & $72$ & $8.49 \pm 0.06$\\ \hline
                $0.7000$ & $63$ & $7.94 \pm 0.06$\\ \hline
                $0.6000$ & $54$ & $7.35 \pm 0.07$\\ \hline
                $0.5000$ & $48$ & $6.93 \pm 0.07$\\ \hline
                $0.4000$ & $42$ & $6.48 \pm 0.08$\\ \hline
                $0.3000$ & $36$ & $6.00 \pm 0.08$\\ \hline
                $0.2000$ & $30$ & $5.48 \pm 0.09$\\ \hline
                $0.1000$ & $25$ & $5.00 \pm 0.10$\\ \hline
                $0.0000$ & $21$ & $4.58 \pm 0.11$\\ \hline
                $-0.1000$ & $17$ & $4.12 \pm 0.12$\\ \hline
                $-0.2000$ & $11$ & $3.3 \pm 0.2$\\ \hline
                $-0.3000$ & $7$ & $2.6 \pm 0.2$\\ \hline
                $-0.4000$ & $3$ & $1.7 \pm 0.3$\\ \hline
                $-0.5000$ & $0$ & $0.0 \pm 1.0$\\ \hline
            \end{tabular}
            \captionof{table}{Результаты измерений для $\lambda~=~5404~\pm~11$~\AA}
        \end{minipage} \hfill
        \begin{minipage}{.25\linewidth}
            \centering
            \begin{tabular}{|c|c|c|}
                \hline

                $U, В$ & $U_i, мВ$ & $\sqrt{U_i}, мВ^{1/2}$\\ \hline
                $1.0000$ & $100$ & $10.00 \pm 0.05$\\ \hline
                $0.9000$ & $84$ & $9.17 \pm 0.05$\\ \hline
                $0.8000$ & $75$ & $8.66 \pm 0.06$\\ \hline
                $0.7000$ & $68$ & $8.25 \pm 0.06$\\ \hline
                $0.6000$ & $58$ & $7.62 \pm 0.07$\\ \hline
                $0.5000$ & $49$ & $7.00 \pm 0.07$\\ \hline
                $0.4000$ & $42$ & $6.48 \pm 0.08$\\ \hline
                $0.3000$ & $36$ & $6.00 \pm 0.08$\\ \hline
                $0.2000$ & $30$ & $5.48 \pm 0.09$\\ \hline
                $0.1000$ & $26$ & $5.099 \pm 0.098$\\ \hline
                $0.0000$ & $24$ & $4.90 \pm 0.10$\\ \hline
                $-0.1000$ & $16$ & $4.00 \pm 0.12$\\ \hline
                $-0.2000$ & $10$ & $3.2 \pm 0.2$\\ \hline
                $-0.3000$ & $7$ & $2.6 \pm 0.2$\\ \hline
                $-0.4000$ & $3$ & $1.7 \pm 0.3$\\ \hline

            \end{tabular}
            \captionof{table}{Результаты измерений для $\lambda~=~5564~\pm~11$~\AA}
        \end{minipage} \hfill
        \begin{minipage}{.30\linewidth}
            \centering
            \begin{tabular}{|c|c|c|}
                \hline

                $U, В$ & $U_i, мВ$ & $\sqrt{U_i}, мВ^{1/2}$\\ \hline
                1.0000 & 104 & $10.20 \pm 0.05$\\ \hline
                0.9000 & 92 & $9.59 \pm 0.05$\\ \hline
                0.8000 & 79 & $8.89 \pm 0.06$\\ \hline
                0.7000 & 72 & $8.49 \pm 0.06$\\ \hline
                0.6000 & 63 & $7.94 \pm 0.06$\\ \hline
                0.5000 & 52 & $7.21 \pm 0.07$\\ \hline
                0.4000 & 45 & $6.71 \pm 0.07$\\ \hline
                0.3000 & 38 & $6.16 \pm 0.08$\\ \hline
                0.2000 & 31 & $5.57 \pm 0.09$\\ \hline
                0.1000 & 25 & $5.00 \pm 0.10$\\ \hline
                0.0000 & 22 & $4.69 \pm 0.11$\\ \hline
                -0.1000 & 15 & $3.87 \pm 0.13$\\ \hline
                -0.2000 & 9 & $3.0 \pm 0.2$\\ \hline
                -0.3000 & 3 & $1.7 \pm 0.3$\\ \hline
                -0.4000 & 0 & $0.0 \pm 1.0$\\ \hline

            \end{tabular}
            \captionof{table}{Результаты измерений для $\lambda = 5737 \pm 10$ \AA}
        \end{minipage}

        \begin{minipage}{.23\linewidth}
            \centering
            \begin{tabular}{|c|c|c|}
                \hline

                $U, В$ & $U_i, мВ$ & $\sqrt{U_i}, мВ^{1/2}$\\ \hline
                1.0000 & 110 & $10.49 \pm 0.05$\\ \hline
                0.9000 & 96 & $9.80 \pm 0.05$\\ \hline
                0.8000 & 84 & $9.17 \pm 0.05$\\ \hline
                0.7000 & 72 & $8.49 \pm 0.06$\\ \hline
                0.6000 & 62 & $7.87 \pm 0.06$\\ \hline
                0.5000 & 53 & $7.28 \pm 0.07$\\ \hline
                0.4000 & 46 & $6.78 \pm 0.07$\\ \hline
                0.3000 & 38 & $6.16 \pm 0.08$\\ \hline
                0.2000 & 31 & $5.57 \pm 0.09$\\ \hline
                0.1000 & 24 & $4.90 \pm 0.10$\\ \hline
                0.0000 & 19 & $4.36 \pm 0.11$\\ \hline
                -0.1000 & 13 & $3.6 \pm 0.1$\\ \hline
                -0.2000 & 6 & $2.4 \pm 0.2$\\ \hline
                -0.3000 & 2 & $1.4 \pm 0.4$\\ \hline
                -0.4000 & 0 & $0.0 \pm 1.0$\\ \hline

            \end{tabular}
            \captionof{table}{Результаты измерений для $\lambda~=~5929~\pm~10$~\AA}
        \end{minipage} \hfill
        \begin{minipage}{.23\linewidth}
            \centering
            \begin{tabular}{|c|c|c|}
                \hline

                $U, В$ & $U_i, мВ$ & $\sqrt{U_i}, мВ^{1/2}$\\ \hline
                1.0000 & 107 & $10.34 \pm 0.05$\\ \hline
                0.8000 & 81 & $9.00 \pm 0.06$\\ \hline
                0.7000 & 69 & $8.31 \pm 0.06$\\ \hline
                0.6000 & 59 & $7.68 \pm 0.07$\\ \hline
                0.5000 & 50 & $7.07 \pm 0.07$\\ \hline
                0.4000 & 40 & $6.32 \pm 0.08$\\ \hline
                0.3000 & 34 & $5.83 \pm 0.09$\\ \hline
                0.2000 & 27 & $5.20 \pm 0.10$\\ \hline
                0.1000 & 19 & $4.36 \pm 0.11$\\ \hline
                0.0000 & 14 & $3.7 \pm 0.1$\\ \hline
                -0.1000 & 8 & $2.8 \pm 0.2$\\ \hline
                -0.2000 & 4 & $2.0 \pm 0.2$\\ \hline
                -0.3000 & 0 & $0.0 \pm 1.0$\\ \hline

            \end{tabular}
            \captionof{table}{Результаты измерений для $\lambda~=~6145~\pm~10$~\AA}
        \end{minipage} \hfill
        \begin{minipage}{.30\linewidth}
            \centering
            \begin{tabular}{|c|c|c|}
                \hline

                $U, В$ & $U_i, мВ$ & $\sqrt{U_i}, мВ^{1/2}$\\ \hline
                1.0000 & 98 & $9.90 \pm 0.05$\\ \hline
                0.9000 & 85 & $9.22 \pm 0.05$\\ \hline
                0.8000 & 72 & $8.49 \pm 0.06$\\ \hline
                0.7000 & 62 & $7.87 \pm 0.06$\\ \hline
                0.6000 & 52 & $7.21 \pm 0.07$\\ \hline
                0.5000 & 43 & $6.56 \pm 0.08$\\ \hline
                0.4000 & 34 & $5.83 \pm 0.09$\\ \hline
                0.3000 & 29 & $5.39 \pm 0.09$\\ \hline
                0.2000 & 21 & $4.58 \pm 0.11$\\ \hline
                0.1000 & 14 & $3.7 \pm 0.1$\\ \hline
                0.0000 & 10 & $3.2 \pm 0.2$\\ \hline
                -0.1000 & 4 & $2.0 \pm 0.2$\\ \hline
                -0.2000 & 0 & $0.0 \pm 1.0$\\ \hline

            \end{tabular}
            \captionof{table}{Результаты измерений для $\lambda = 6391 \pm 10$ \AA}
            \label{}
        \end{minipage}

        \begin{minipage}{.4\linewidth}
            \centering
            \begin{tabular}{|c|c|c|}
                \hline

                $U, В$ & $U_i, мВ$ & $\sqrt{U_i}, мВ^{1/2}$\\ \hline
                1.0000 & 88 & $9.38 \pm 0.05$\\ \hline
                0.9000 & 74 & $8.60 \pm 0.06$\\ \hline
                0.8000 & 63 & $7.94 \pm 0.06$\\ \hline
                0.7000 & 53 & $7.28 \pm 0.07$\\ \hline
                0.6000 & 44 & $6.63 \pm 0.08$\\ \hline
                0.5000 & 34 & $5.83 \pm 0.09$\\ \hline
                0.4000 & 26 & $5.099 \pm 0.098$\\ \hline
                0.3000 & 21 & $4.58 \pm 0.11$\\ \hline
                0.2000 & 15 & $3.87 \pm 0.13$\\ \hline
                0.1000 & 9 & $3.0 \pm 0.2$\\ \hline
                0.0000 & 6 & $2.4 \pm 0.2$\\ \hline
                -0.1000 & 0 & $0.0 \pm 1.0$\\ \hline

            \end{tabular}
            \captionof{table}{Результаты измерений для $\lambda~=~6673~\pm~9$~\AA}
            \label{}
        \end{minipage} \hfill
        \begin{minipage}{.4\linewidth}
            \centering
            \begin{tabular}{|c|c|c|}
                \hline

                $U, В$ & $U_i, мВ$ & $\sqrt{U_i}, мВ^{1/2}$\\ \hline
                1.0000 & 65 & $8.06 \pm 0.06$\\ \hline
                0.9000 & 51 & $7.14 \pm 0.07$\\ \hline
                0.8000 & 40 & $6.32 \pm 0.08$\\ \hline
                0.7000 & 33 & $5.74 \pm 0.09$\\ \hline
                0.6000 & 25 & $5.00 \pm 0.10$\\ \hline
                0.5000 & 20 & $4.47 \pm 0.11$\\ \hline
                0.4000 & 14 & $3.7 \pm 0.1$\\ \hline
                0.3000 & 11 & $3.3 \pm 0.2$\\ \hline
                0.2000 & 6 & $2.4 \pm 0.2$\\ \hline
                0.1000 & 3 & $1.7 \pm 0.3$\\ \hline
                0.0280 & 0 & $0.0 \pm 1.0$\\ \hline

            \end{tabular}
            \captionof{table}{Результаты измерений для $\lambda~=~7025~\pm~9$~\AA}
            \label{}
        \end{minipage}




\end{document}

