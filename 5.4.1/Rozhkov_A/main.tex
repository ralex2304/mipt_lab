\documentclass[a4paper, 12pt]{article}
\usepackage[a4paper,top=1.5cm, bottom=1.5cm, left=1cm, right=1cm]{geometry}
\usepackage{cmap}					% поиск в PDF
\usepackage{mathtext} 				% русские буквы в формулах
\usepackage[T2A]{fontenc}			% кодировка
\usepackage[utf8]{inputenc}			% кодировка исходного текста
\usepackage[english,russian]{babel}	% локализация и переносы

\usepackage{amsmath,amssymb}
\usepackage{indentfirst}
\usepackage{longtable}
\usepackage{graphicx}
\usepackage{array}
\usepackage{float}

\usepackage{floatflt}
\usepackage{wrapfig}
\usepackage{siunitx} % Required for alignment
\usepackage{subfig}
\usepackage{multirow}
\usepackage{rotating}
\usepackage{caption}

\graphicspath{{.}}


\title{\begin{center}Лабораторная работа №5.4.1\end{center}
Определение энергии $\alpha$-частиц по величине их пробега в воздухе}
\author{Рожков А. В.}
\date{\today}

\begin{document}
    \pagenumbering{gobble}
    \maketitle
    \newpage
    \pagenumbering{arabic}

    \textbf{Цель работы:} измерить пробег $\alpha$-частиц в воздухе тремя способами:
    с помощью торцевого счётчика Гейгера, сцинтилляционного счетчика и ионизационной камеры,
    - по полученным величинам определяется энергия частиц

    \textbf{В работе используются:} счётчик Гейгера, сцинтилляционный счетчик, ионизационная камера.

    \section{Теоретические сведения}

        При $\alpha$-распаде исходное родительское ядро испускает ядро гелия и превращается в дочернее ядро, число протонов и число нейтронов уменьшается на две единицы. Функциональная связь между энергией $\alpha$-частицы $E$ и периодом полураспада радиоактивного ядра $T_{1/2}$ хорошо описывается формулой
        \begin{equation*}
             \lg T_{1/2} = \frac{a}{\sqrt{E}} + b.
        \end{equation*}

        Экспоненциальный характер этого процесса возникает вследствие экспоненциального затухания волновой функции в области под барьером, где потенциальная энергия больше энергии частицы.


        Экспериментально энергию $\alpha$-частиц удобно определять по величине их пробега в веществе.
        Для описания связи между энергией $\alpha$-частицы и ее пробегом пользуются эмпирическими соотношениями. В диапазоне энергий $\alpha$-частиц от $4$ до $9$ МэВ эта связь хорошо описывается выражением
        \begin{equation}
            R = 0,32 E^{3/2},
        \end{equation}
        где пробег $\alpha$-частиц в воздухе $R$ (при 15 $^\circ C$ и атмосферном давлении) выражается в см, а энергия частицы $E$ в МэВ.

        \begin{figure}[h!]
            \centering
            \includegraphics[width = 0.5\linewidth]{img/dn_dx.png}
            \caption{Зависимость числа $\alpha$-частиц от глубины их проникновения в вещество}
            \label{}
        \end{figure}

        При малых глубинах число частиц не меняется с расстоянием. В конце пути это число не сразу обрывается до нуля, а приближается к нему постепенно. Как видно из кривой $dN/dx$, большая часть $\alpha$-частиц останавливается в узкой области, расположенной около некоторого значения $x$, которое называется средним пробегом $R_{\text{ср}}$. Иногда вместо $R_{\text{ср}}$ измеряются экстраполированное значение $R_{\text{э}}$.

        В силу размытия и смещения брэгговского пика из-за угловой расходимости пучков частиц, лучшей оценкой пробега оказывается экстраполированный пробег.

    \section{Экспериментальная установка}

        \subsection{Счётчик Гейгера}

            Для определения пробега $\alpha$-частиц с помощью счетчика радиоактивный источник помещается на дно стальной цилиндрической бомбы, в которой может перемещаться торцевой счетчик Гейгера. Его чувствительный объем отделен от наружной среды тонким слюдяным окошком, сквозь которое могут проходить $\alpha$-частицы.

            Импульсы, возникающие в счетчике, усиливаются и регистрируются пересчетной схемой. Путь частиц в воздухе зависит от расстояния между источником и счетчиком. Перемещение счетчика производится путем вращения гайки, находящейся на крышке бомбы. Расстояние между счетчиком и препаратом измеряется по шкале, нанесенной на держатель счетчика.

            \begin{figure}[h!]
                \centering
                \includegraphics[width = 0.3\linewidth]{img/geiger.png}
                \caption{Счётчик Гейгера}
                \label{}
            \end{figure}

        \subsection{Сцинтилляционный счётчик}

            Установка состоит из цилиндрической камеры, на дне которой находится исследуемый препарат. Камера герметично закрыта стеклянной пластинкой, на которую с внутренней стороны нанесен слой люминофора. С наружной стороны к стеклу прижат фотокатод фотоумножителя. Оптический контакт ФЭУ-стекло обеспечивается тонким слоем вазелинового масла.

            Сигналы с фотоумножителя через усилитель поступают на пересчетную установку. Расстояние между препаратом и люминофором составляет 9 см, так что $\alpha$-частицы не могут достигнуть люминофора при обычном давлении. Определение пробега сводится к измерению зависимости интенсивности счета от давления в камере.

            \begin{figure}[h!]
                \centering
                \includegraphics[width = 0.3\linewidth]{img/scint.png}
                \caption{Установка для измерения пробега $\alpha$-частиц с помощью сцинтилляционного счетчика}
                \label{}
            \end{figure}

        \subsection{Ионизационная камера}

            Ионизационная камера -- прибор для количественного измерения ионизации, произведенной заряженными частицами при прохождении через газ. Камера представляет собой наполненный газом сосуд с двумя электродами. Сферическая стенка прибора служит одним из электродов, второй электрод вводится в газ через изолирующую пробку. К электродам подводится постоянное напряжение от источника ЭДС. Заполняющий сосуд газ сам по себе не проводит электрический ток, возникает он только при прохождении быстрой заряженной частицы, которая рождает в газе на своем пути ионы. Поместим на торец внутреннего электрода источник ионизирующего излучения, заполним объем камеры воздухом.

            Прохождение тока через камеру регистрируется посредством измерения напряжения на включенном в цепь камеры сопротивлении $R$. При небольших давлениях газа $\alpha$-частицы передают часть энергии стенкам камеры. По достижении давления $P_0$ все они заканчивают свой пробег внутри газа, и дальнейшее возрастание тока прекращается. Для определения давления $P_0$ чаще всего пользуются методом экстраполяции, продолжая наклонный и горизонтальный участки кривой до пересечения. Найденный таким образом пробег затем должен быть приведен к нормальному давлению $760~торр$ и температуре $15^oC$.

            \begin{figure}[h!]
                \centering
                \includegraphics[width = 0.4\linewidth]{img/ion.png}
                \caption{Установка для измерения пробега $\alpha$-частиц с помощью ионизационной камеры}
                \label{}
            \end{figure}

    \section{Ход работы}

        В начале и в конце проведения измерений было записано атмосферное давление с 2 барометров:

        \begin{table}[!ht]
            \centering
            \begin{tabular}{|c|c|c|}
                \hline

                Барометр & До & После\\ \hline
                1 & $727.9 \pm 0.1$$~торр$ & $727.8 \pm 0.1$$~торр$\\ \hline
                2 & $97.7 \pm 0.1$$~кПа$ & $97.8 \pm 0.1$$~кПа$\\ \hline

            \end{tabular}
            \caption{Атмосферное давление до и после проведения измерений}
            \label{}
        \end{table}

        В дальнейшем в работе используется среднее значение $P_{атм} = 730 \pm 2~торр$. Температуру примем равной $22^oC$

        \subsection{Измерение длины пробега $\alpha$-частиц с помощью сцинтилляционного счётчика}

            Построим график зависимости $N(P)$ по полученным данным, где $N = N_0 / t$, $\sigma_N = \sqrt{N_0} / t$:

            \begin{figure}[h!]
                \centering
                \includegraphics[width = 0.75\linewidth]{img/scint_exp.png}
                \caption{График зависимости $N(P)$}
                \label{}
            \end{figure}

            Аппроксимировали функцией $y = k*x + b$. Значения коэффициентов: $k = -1.61 \pm 0.05~с^{-1}торр^{-1}$; $b = 403 \pm 5~с^{-1}$

            Находим $N_{ср} = \frac{N_{max} + N_{min}}{2}$, где $P_{min} = 0~торр$, $P_{max} = 350 \pm 50~торр$. По нему по аппроксимированной кривой находим $P_{ср}$. Погрешность по формуле погрешности сложного измерения

            Полученные значения:
            $$
                P_{\text{э}} = \frac{-b}{k} = 250 \pm 8~торр
            $$
            $$
                P_{\text{ср}} = \frac{N_{ср} - b}{k} = 130 \pm 20~торр
            $$

            Было получено давление, при котором длина свободного пробега равна расстоянию от источника для люминофора $L = 9~см$: $R = L \frac{P}{P_0} \frac{T_0}{T}$

            $$
                R_{э} = 2.88 \pm 0.09~см, R'_{э} = \rho R_{э} = 1.225 ~кг/м^3 \cdot R_{э} = 3.53 \pm 0.11 \cdot 10^{-3} ~г/см^2
            $$
            $$
                R_{ср} = 1.6 \pm 0.2~см, R'_{ср} = \rho R_{ср} = 1.225 ~кг/м^3 \cdot R_{ср} = 1.9 \pm 0.2 \cdot 10^{-3} ~г/см^2
            $$
            Экстраполированной длине свободного пробега соответствует значение энергии $E_{э} = 4.3 \pm 0.1~МэВ$. Полученное значение отличается от табличного $5.15~МэВ$.

            Считая, что эффективность счёта $\alpha$-частиц равно $100\%$, оценим по известному периоду полураспада количество вещества в препарате. Телесный угол, под которым виден источник равен $0.04~ср$

            $$
            \nu = A \frac{T_{1/2}}{ln2} = N_{max} * \frac{4 \pi}{0.04~ср} \cdot \frac{T_{1/2}}{ln2} = 370~с^{-1} \frac{4\pi}{0.04~ср} \cdot \frac{24400~лет}{ln2} = 2.1 * 10^{-7}~моль
            $$
            $$
                m = M \nu = 273~г/моль * 2.1 * 10^{-7}~моль = 58~мкг
            $$

        \subsection{Измерение длины пробега $\alpha$-частиц с помощью ионизационной камеры}

            Построим график зависимости $I(P)$ по полученным данным:

            \begin{figure}[h!]
                \centering
                \includegraphics[width = 0.75\linewidth]{img/ion_exp.png}
                \caption{График зависимости $I(P)$}
                \label{}
            \end{figure}

            С помощью аппроксимационных прямых определяем давление, при котором начинается область насыщения. Это давление соответствующее точке пересечения прямых. Погрешности по формуле погрешности сложного измерения. Коэффициенты прямых:
            $$
                k_1 = 1.756 \pm 0.011~пА/торр \text{~~~~} b_1 = -86 \pm 4~пА
            $$
            $$
                k_2 = -0.24 \pm 0.02~пА/торр \text{~~~~} b_2 = 960 \pm 10~пА
            $$

            $$
                P = \frac{b_1 - b_2}{k_2 - k_1} = 523 \pm 10~торр
            $$

            Это давление, при котором длина свободного пробега равна расстоянию между внутренним и внешним электродами $L = (10 - 0.5)/2 = 4.75 ~см$ (10 см – диаметр внешнего диска, 0.5 см – внутреннего). Пересчитаем длину свободного пробега для нормальных условий:

            $$
                R_{э} = L \frac{P}{P_0} = 3.19 \pm 0.06~см, R' = \rho R =  1.225~кг/м^3 \cdot R_{э} = 3.91 \pm 0.07 \cdot 10^{-3}~г/см^2
            $$
            Этой длине свободного пробега соответствует значение энергии $E = 4.63 \pm 0.08~МэВ$.

            Для данной энергии рассчитаем толщину бумажного листа, достаточного, чтобы остановить $\alpha$-частицу.

            $$
                d = R_{бумага} = \frac{R'}{\rho_{бумага}} = \frac{3.91*10^{-3}~г/см^2}{0.8~г/см^3} = 0.05~мм
            $$

    \section{Заключение}

        Двумя различными способами был измерен свободный пробег в воздухе $\alpha$-частиц с энергией 5.15 МэВ. В качестве источника радиоактивных частиц был использован $^{239}Pu$.

        В результате экспериментов были получены следующие значения энергии $\alpha$-частиц: с помощью сцинтилляционного счётчика $E_{э} = 4.3 \pm 0.1~МэВ$, с помощью ионизационной камеры $E_{э} = 4.63 \pm 0.08~МэВ$.

        Полученные значения является заниженными по сравнению с теоретическим ($5.15~МэВ$) по следующим причинам: источник частиц покрыт слюдяной пленкой, что приводит к замедлению $\alpha$-частиц; пучки частиц обладают конечными размерами, что приводит к угловой расходимости и заметно искажает брэгговский пик, из-за чего зависимости являются более размытыми.

        Проведена оценка количества вещества и массы $^{125}Pu$ в установке с сцинтилляционным счётчиком -- $\nu = 2.1*10^{-7}~моль$, $m = 58~мкг$.

        Также для источника с такой энергией $\alpha$-частиц достаточно листа бумаги толщиной примерно $0.05~мм$, чтобы их остановить.

    \section{Приложение}


        \begin{table}[!ht]
            \centering
            \begin{tabular}{|c|c|c|c|c|}
                \hline

                $P_{изм}, торр$ & $t, с$ & $N_0$ & $N, с^{-1}$ & $P, торр$\\ \hline
                $-710$ & $10$ & $3697$ & $369.7$ & $20 \pm 2$\\ \hline
                $-700$ & $10$ & $3406$ & $340.6$ & $30 \pm 2$\\ \hline
                $-690$ & $10$ & $3337$ & $333.7$ & $40 \pm 2$\\ \hline
                $-680$ & $10$ & $3157$ & $315.7$ & $50 \pm 2$\\ \hline
                $-670$ & $10$ & $3035$ & $303.5$ & $60 \pm 2$\\ \hline
                $-660$ & $10$ & $2901$ & $290.1$ & $70 \pm 2$\\ \hline
                $-650$ & $10$ & $2725$ & $272.5$ & $80 \pm 2$\\ \hline
                $-640$ & $10$ & $2569$ & $256.9$ & $90 \pm 2$\\ \hline
                $-630$ & $16$ & $4054$ & $253.375$ & $100 \pm 2$\\ \hline
                $-620$ & $20$ & $4544$ & $227.2$ & $110 \pm 2$\\ \hline
                $-610$ & $15$ & $3208$ & $213.867$ & $120 \pm 2$\\ \hline
                $-600$ & $20$ & $3779$ & $188.95$ & $130 \pm 2$\\ \hline
                $-590$ & $15$ & $2741$ & $182.733$ & $140 \pm 2$\\ \hline
                $-580$ & $20$ & $2960$ & $148.0$ & $150 \pm 2$\\ \hline
                $-570$ & $20$ & $2793$ & $139.65$ & $160 \pm 2$\\ \hline
                $-550$ & $30$ & $2988$ & $99.6$ & $180 \pm 2$\\ \hline
                $-530$ & $40$ & $2668$ & $66.7$ & $200 \pm 2$\\ \hline
                $-510$ & $70$ & $2586$ & $36.943$ & $220 \pm 2$\\ \hline
                $-490$ & $100$ & $1777$ & $17.770$ & $240 \pm 2$\\ \hline
                $-470$ & $100$ & $491$ & $4.910$ & $260 \pm 2$\\ \hline

            \end{tabular}
            \caption{Результаты измерений для эксперимента со сцинтилляционным счётчиком}
            \label{}
        \end{table}


        \begin{table}[!ht]
            \centering
            \begin{tabular}{|c|c|c|}
                \hline

                $P_{изм}, торр$ & $P, торр$ & $I, пА$\\ \hline
                $-700$ & $30 \pm 2$ & $28$\\ \hline
                $-650$ & $80 \pm 2$ & $98$\\ \hline
                $-600$ & $130 \pm 2$ & $168$\\ \hline
                $-550$ & $180 \pm 2$ & $243$\\ \hline
                $-500$ & $230 \pm 2$ & $324$\\ \hline
                $-475$ & $255 \pm 2$ & $363$\\ \hline
                $-450$ & $280 \pm 2$ & $405$\\ \hline
                $-425$ & $305 \pm 2$ & $447$\\ \hline
                $-400$ & $330 \pm 2$ & $491$\\ \hline
                $-375$ & $355 \pm 2$ & $535$\\ \hline
                $-350$ & $380 \pm 2$ & $580$\\ \hline
                $-325$ & $405 \pm 2$ & $627$\\ \hline
                $-300$ & $430 \pm 2$ & $672$\\ \hline
                $-275$ & $455 \pm 2$ & $715$\\ \hline
                $-250$ & $480 \pm 2$ & $758$\\ \hline
                $-225$ & $505 \pm 2$ & $787$\\ \hline
                $-200$ & $530 \pm 2$ & $810$\\ \hline
                $-175$ & $555 \pm 2$ & $820$\\ \hline
                $-150$ & $580 \pm 2$ & $820$\\ \hline
                $-125$ & $605 \pm 2$ & $810$\\ \hline
                $-100$ & $630 \pm 2$ & $806$\\ \hline
                $-75$ & $655 \pm 2$ & $795$\\ \hline
                $-50$ & $680 \pm 2$ & $795$\\ \hline
                $-25$ & $705 \pm 2$ & $790$\\ \hline
                $0$ & $730 \pm 2$ & $780$\\ \hline

            \end{tabular}
            \caption{Результаты измерений для эксперимента с ионизационной камерой}
            \label{}
        \end{table}

\end{document}

