\documentclass[a4paper, 12pt]{article}
\usepackage[a4paper,top=1.5cm, bottom=1.5cm, left=1cm, right=1cm]{geometry}
\usepackage{cmap}					% поиск в PDF
\usepackage{mathtext} 				% русские буквы в формулах
\usepackage[T2A]{fontenc}			% кодировка
\usepackage[utf8]{inputenc}			% кодировка исходного текста
\usepackage[english,russian]{babel}	% локализация и переносы

\usepackage{amsmath,amssymb}
\usepackage{indentfirst}
\usepackage{longtable}
\usepackage{graphicx}
\usepackage{array}
\usepackage{float}

\usepackage{floatflt}
\usepackage{wrapfig}
\usepackage{siunitx} % Required for alignment
\usepackage{subfig}
\usepackage{multirow}
\usepackage{rotating}
\usepackage{caption}

\graphicspath{{.}}

\title{\begin{center}Лабораторная работа №5.8.1\end{center}
Определения постоянных Стефана-Больцмана и Планка из анализа теплового излучения накаленного тела}
\author{Рожков А. В.}
\date{\today}

\begin{document}
    \pagenumbering{gobble}
    \maketitle
    \newpage
    \pagenumbering{arabic}

    \textbf{Цель работы:}  при помощи модели абсолютно чёрного тела провести измерения температуры оптическим пирометром с исчезающей нитью и термопарой; определить постоянные Планка и Стефана-Больцмана.

    \textbf{В работе используются:} оптический пирометр, модель абсолютно черного тела (АЧТ), лампа накаливания (с вольфрамовой нитью), керамическая трубка с кольцами из разных материалов, неоновая лампа, блок питания, цифровые вольтметры и амперметр.

    \section{Теоретические сведения}

        Связь между испускательной и поглощательной способностями любого тела устанавливает закон Кирхгофа:

        \begin{equation*}
            \frac{r_{\lambda, T}}{a_{\lambda, T}} = f(\lambda, T)
        \end{equation*}

        Так как для абсолютно чёрного тела $a_{\lambda, T} = 1$, для всех остальных (нечёрных) тел закон излучения Кирхгофа записывается в виде:

        \begin{equation*}
            r_{\lambda, T} = a_{\lambda, T} r_{\lambda, T}^{ачт}
        \end{equation*}

        Для измерения температуры разогретых тел, удалённых от наблюдателя, применяют методы оптической пирометрии, основанные на использовании зависимости испускательной способности исследуемого тела от температуры. Различают три температуры, функционально связанные с истинной термодинамической температурой и излучательной способностью тела: радиационную $T_{rad}$, цветовую $T_{col}$ и яркостную $T_{br}$.

        В работе измеряется яркостная температура -- температура абсолютно чёрного тела, при которой его спектральная испускательная способность равна спектральной испускательной способности исследуемого тела при той же длине волны. Измерение яркостной температуры раскалённого тела производится при помощи оптического пирометра с исчезающей нитью, основанного на визуальном сравнении яркости раскалённой нити с яркостью изображения исследуемого тела. Равенство видимых яркостей, наблюдаемых через монохроматический светофильтр ($\lambda = 650$ нм), фиксируется по исчезновению изображения нити на фоне раскаленного тела. Яркостный метод измерения температуры основан, в соответствии с формулой Планка, на зависимости испускательной способности абсолютно чёрного тела от температуры и длины волны.

        \[ r^{\text{ачт}}_{\lambda, T} = \frac{2 \pi c^2 h}{\lambda^5} \frac{1}{\exp[hc/(\lambda k_{Б} T)] - 1} \]

        Шкалу прибора, измеряющего ток через нить, предварительно градуируют по абсолютно черному телу, термодинамическую температуру которого измеряют с помощью термопары. Если тело, температуру которого определяют, излучает как абсолютно черное тело, то, мы тем самым можем с помощью пирометра найти его температуру. Если же тело излучает иначе, то определенное значение температуры является яркостной температурой. Яркостная температура тела всегда ниже его термодинамической температуры.

        Среди нечерных тел выделяются так называемые серые тела, для которых характер распределения излучения совершенно подобен спектру абсолютно черного тела, но излучение ослаблено по сравнению с ним в $\varepsilon_T$ раз для любой длины волны при данной температуре тела $T$.

        Если предположить, что тело излучает как серое тело, то выражения для энергии излучения можно записать как

        \begin{equation*}
            W = \varepsilon_T S \sigma T^4, ~~ \sigma = \frac{2 \pi^5 k_{Б}^4}{15 c^2 h^3} = 5,67 \cdot 10^{-12} \frac{\text{Вт}}{\text{см}^2 \cdot K^4}
        \end{equation*}

        с учётом того, что температура тела намного больше температуры окружающей среды, и мощностью внешних потерь можно пренебречь.

        Из этого выражения можно экспериментально получить значение постоянной Планка:

        \begin{equation*}
            h = \sqrt[3]{\frac{2\pi^5k_{Б}^4}{15c^2\sigma}}
        \end{equation*}

        Положение максимума спектрального распределения определяется законом смещения Вина:

        \begin{equation*}
            \lambda_{max}T = b = 2.897 * 10^{-3}~м\cdotК
        \end{equation*}

        \begin{figure}[ht!]
            \begin{center}
                \includegraphics[width = 0.4\textwidth]{img/wien_law.png}
                \caption{Распределение энергии в спектре излучения абсолютно чёрного тела при различных температурах}
                \label{img:wien_law}
            \end{center}
        \end{figure}

        Для вольфрама имеет место селективность излучения в коротковолновом диапазоне. Оказывается, что излучение в видимой области спектра значительно больше, чем это следует из распределения Планка для АЧТ. Разницу между спектрами при температуре $2450~K$ иллюстрирует рис. \ref{img:w_select}.

        \begin{figure}[ht!]
            \begin{center}
                \includegraphics[width = 0.4\textwidth]{img/w_select.png}
                \caption{Распределение энергии в спектре излучения: 1 -- АЧТ; 2 -- вольфрам. Температура $2450~K$}
                \label{img:w_select}
            \end{center}
        \end{figure}

    \section{Экспериментальная установка}

        Экспериментальная установка состоит из оптического пирометра 9, модели абсолютно черного тела (АЧТ), трех исследуемых образцов \\ ($18, 19, 20$), блока питания (1) и вольтметров.

        Оптический пирометр представляет собой зрительную трубу, внутри которой имеется накаливаемая нить, расположенная в плоскости изображения исследуемого раскаленного тела, а также темно-красный светофильтр ($\lambda = 650$ нм). Через окуляр одновременно наблюдается изображение исследуемого тела и раскаленной нити.

        Модель АЧТ представляет собой керамическую трубку, закрытую с одного конца и окруженную для теплоизоляции внешним кожухом. Нагрев трубки осуществляется намотанной на ней нихромовой спиралью, питаемой от источника тока. Полость трубки и особенно ее дно излучают практически как абсолютно черное тело. Температура модели АЧТ измеряется хромельалюмелевой термопарой.

        В работе исследуются три образца. Один образец выполнен в виде керамической трубки с набором колец. Другой исследуемый образец -- вольфрамовая нить электрической лампочки. По показаниям вольтметра и амперметра можно определить мощность, потребляемую нитью лампочки.

        \begin{figure}[ht!]
            \begin{center}
                \includegraphics[width = 0.5\textwidth]{img/setup.png}
                \caption{Схема экспериментальной установки: 1 -- блок питания; 2 -- тумблер включения питания пирометра и образцов; 3 -- тумблер нагрева нити пирометра: <<Быстро>> -- вверх, <<Медленно>> -- вниз; 4 -- кнопка <<Нагрев нити>>; 5 -- кнопка <<охлаждение нити>>; 6 -- тумблер переключения образцов; 7 -- регулятор мощности нагрева образцов; 8 -- окуляр пирометра; 9 -- корпус пирометра; 10 -- объектив пирометра; 11 -- переключение диапазонов: 700-1200 C -- вниз, 1200-2000 C -- вверх; 12 -- ручка перемещения красного светофильтра; 13 -- регулировочный винт; 14 -- вольтметр (напряжение на лампе накаливания); 15 -- амперметр (ток через образцы); 16 -- вольтметр в цепи термопары; 17 -- модель АЧТ; 18 -- трубка с кольцами из материалов с разной излучательной способностью; 19 -- лампа накаливания; 20 -- неоновая лампочка}
                \label{setup1}
            \end{center}
        \end{figure}

        Для обработки данных необходима также зависимость между яркостной и термодинамической температурами вольфрама, приведённая на рис. \ref{img:wolfram_brightness}.

        \begin{figure}[ht!]
            \begin{center}
                \includegraphics[width = 0.5\textwidth]{img/wolfram_brightness.png}
                \caption{Зависимость между яркостной и термодинамической температурами вольфрама}
                \label{img:wolfram_brightness}
            \end{center}
        \end{figure}

    \newpage

    \section{Ход работы}

        \subsection{Проверка работы пирометра}

            Определим исправность пирометра при помощи измерения температуры модели абсолютно чёрного тела. Температура измеряется при помощи хромель-алюмелевой термопары с постоянной $41~мкВ/^oC$.

            Провели 4 измерения (2 раза отслеживали совпадение при повышении температуры нити пирометра и 2 при понижении). Результаты в таблице \ref{tab:acht}.

            \begin{table}[!ht]
                \centering
                \begin{tabular}{|c|c|c|}
                    \hline

                    $T_{термопара}, ^oC$ & $T_{пирометр}, ^oC$ & $\Delta, \%$ \\ \hline
                    $936.1 \pm 0.2$ & $952 \pm 1$ & $1.67 \pm 0.11$\\ \hline
                    $940.0 \pm 0.2$ & $962 \pm 1$ & $2.29 \pm 0.10$\\ \hline
                    $943.2 \pm 0.2$ & $960 \pm 1$ & $1.75 \pm 0.11$\\ \hline
                    $947.3 \pm 0.2$ & $970 \pm 1$ & $2.34 \pm 0.10$\\ \hline

                \end{tabular}
                \caption{Измерение температуры абсолютно чёрного тела}
                \label{tab:acht}
            \end{table}

            Убедились, что разница не превышает $5\%$

        \subsection{Измерение яркостной температуры накалённых тел}

            Измерили яркостные температуры двух нагретых колец, находящихся на одной керамической трубке. Получили $765~^oC$ и $775~^oC$. Разница яркостных температур при одной и той же термодинамической объясняется различием спектральных светимостей материалов.

        \subsection{Проверка закона Стефана-Больцмана}

            Измерялась яркостная температура вольфрамовой нити лампы накаливания. Также регистрировались ток и напряжение на нити для определения электрической мощности.

            Сводная таблица результатов для данного пункта находится в приложении

            \subsubsection{Определение термодинамической температуры нити}

                Аппроксимируем зависимость $T = f(T_{ярк})$ линейной функцией по 12 точкам графика на рис. \ref{img:wolfram_brightness}.

                Получившаяся прямая на рис. \ref{plot:wolfram_brightness_table}. $k = 1.068 \pm 0.006$; $b = (-45 \pm 9)~K$.

                \begin{figure}[ht!]
                    \begin{center}
                        \includegraphics[width = 0.5\textwidth]{img/T_bright_table.png}
                        \caption{Аппроксимация зависимости между яркостной и термодинамической температурами вольфрама}
                        \label{plot:wolfram_brightness_table}
                    \end{center}
                \end{figure}

                При помощи данной аппроксимации определили термодинамическую температуру для каждого измерения яркостной.

            \subsubsection{Зависимость мощности на нити от термодинамической температуры}

                График $W = f(T)$ представлен на рис. \ref{plot:W_T}.

                \begin{figure}[ht!]
                    \begin{center}
                        \includegraphics[width = 0.5\textwidth]{img/W_T.png}
                        \caption{График зависимости электрической мощности на нити лампы от термодинамической температуры}
                        \label{plot:W_T}
                    \end{center}
                \end{figure}

            \subsubsection{Коэффициенты излучения для вольфрама}

                Аппроксимируем линейной функцией таблицу с коэффициентами излучения для разных температур (таблица \ref{tab:wolfram_izl}). Результат на рис. \ref{plot:Eps_T}

                \begin{table}[!ht]
                    \centering
                    \begin{tabular}{|c|c|}
                        \hline

                        $T,K$ & $\varepsilon_T$\\ \hline
                        $800$ & $0.067$\\ \hline
                        $900$ & $0.081$\\ \hline
                        $1000$ & $0.105$\\ \hline
                        $1100$ & $0.119$\\ \hline
                        $1200$ & $0.133$\\ \hline
                        $1300$ & $0.144$\\ \hline
                        $1400$ & $0.164$\\ \hline
                        $1500$ & $0.179$\\ \hline
                        $1600$ & $0.195$\\ \hline
                        $1700$ & $0.209$\\ \hline
                        $1800$ & $0.223$\\ \hline
                        $1900$ & $0.236$\\ \hline
                        $2000$ & $0.249$\\ \hline

                    \end{tabular}
                    \caption{Коэффициенты излучения вольфрама}
                    \label{tab:wolfram_izl}
                \end{table}

                \begin{figure}[ht!]
                    \begin{center}
                        \includegraphics[width = 0.5\textwidth]{img/eps_T.png}
                        \caption{Аппроксимация зависимости коэффициента излучения вольфрама от температуры}
                        \label{plot:Eps_T}
                    \end{center}
                \end{figure}

                Получили $k = (152 \pm 2)*10^{-6}~K^{-1}$; $b = -0.051 \pm 0.003$

            \subsubsection{Проверка закона Стефана-Больцмана}

                Построим в логарифмическом масштабе график зависимости

                \begin{align*}
                    W &= \varepsilon_T B T^n & lnW - ln(\varepsilon_T B) &= n lnT,
                \end{align*} где $B = S \sigma$, $S$ - эффективная площадь излучающей поверхности лампы при температуре более $1500^oC$.

                $\varepsilon_T$ определяем при помощи полученной в предыдущем пункте аппроксимации.

                \begin{figure}[ht!]
                    \begin{center}
                        \includegraphics[width = 0.5\textwidth]{img/ln_W_ln_T.png}
                        \caption{График зависимости $W = f(T)$ в логарифмическом масштабе}
                        \label{plot:ln_W_ln_T}
                    \end{center}
                \end{figure}

                График представляет собой прямую пропорциональность с коэффициентом $n \approx 3$. Теоретически значение должно быть равно $4$.

            \subsection{Определение величин постоянных Стефана-Больцмана и Планка}

                При помощи измерений при температурах больших $1700~K$ определим искомые постоянные по формулам:

                \begin{align*}
                    \sigma &= \frac{W}{\varepsilon_T S T^4} & h &= \sqrt[3]{\frac{2\pi^5k_{Б}^4}{15c^2\sigma}}
                \end{align*}

                Результаты в таблице \ref{tab:sigma_h}.

                \begin{table}[!ht]
                    \centering
                    \begin{tabular}{|c|c|c|}
                        \hline

                        $T, K$ & $\sigma*10^{-12}, \frac{Вт}{см^2 \cdot K^4}$ & $h*10^{-34}, Дж \cdot с$\\ \hline
                        $1780 \pm 10$ & $4.2 \pm 0.1$ & $7.3 \pm 0.2$\\ \hline
                        $1850 \pm 10$ & $4.12 \pm 0.13$ & $7.4 \pm 0.2$\\ \hline
                        $1990 \pm 10$ & $4.00 \pm 0.12$ & $7.4 \pm 0.2$\\ \hline
                        $2070 \pm 10$ & $3.64 \pm 0.11$ & $7.7 \pm 0.2$\\ \hline
                        $2220 \pm 20$ & $2.94 \pm 0.08$ & $8.2 \pm 0.2$\\ \hline
                        $2310 \pm 20$ & $3.05 \pm 0.08$ & $8.1 \pm 0.2$\\ \hline
                        Среднее & $3.7 \pm 0.3$ & $7.7 \pm 0.3$ \\ \hline

                    \end{tabular}
                    \caption{Экспериментальные значения постоянных Стефана-Больцмана и Планка}
                    \label{tab:sigma_h}
                \end{table}

                Как видим по порядку величины значения совпали с известными ($\sigma = 5.67*10^{-12} \frac{Вт}{см^2 \cdot K^4}$, $h = 6.26*10^{-34}~Дж \cdot K$)

        \subsection{Измерение "яркостной температуры" неоновой лампочки}

            При измерении пирометром кажется, что лампочка нагрета до температуры $819~^oC$, хотя её термодинамическая температура примерно равна комнатной. Такой эффект объясняется тем, что неоновая лампочка не является абсолютно чёрным или серым телом. Природа свечения совсем иная: электроны излучают фотоны при переходах между энергетическими уровнями

    \section{Вывод}

        В данной работе проверили работоспособность оптического пирометра на модели АЧТ (полученные значения отличаются от измеренных термопарой не более чем на $2.5\%$).

        Для двух колец, изготовленных из различных материалов, зафиксировано различие яркостных температур при $\lambda = 650~нм$ при одинаковых термодинамических температурах, так как различные материалы могут иметь различные зависимости спектральной светимости от длины волны.

        Получить степень $n = 4$ для закона Стефана-Больцмана не удалось (результат примерно равен $3$). Это можно объяснить ошибкой в визуальном определении яркости. Резкость изображения была удовлетворительной только в центре области видимости. Также качество изображения ухудшалось с увеличением яркости и точно определять температуру было сложнее, что видно и по разбросу точек на графике.

        Экспериментально определённые постоянные Стефана-Больцмана и Планка по порядку величины совпадают с известными значениями. Получить более точный результат помешали вышеуказанные причины и то, что в модели не учтены тепловые потери от вольфрамовой нити. Предполагалось, что вся мощность уходит в излучение.

        Наконец, было определено, что "яркостная температура" неоновой лампочки не совпадает с термодинамической, так как лампочка не является моделью АЧТ и свечение имеет совершенно другую природу.

    \section{Приложение}

        \begin{table}[!ht]
            \centering
            \begin{tabular}{|c|c|c|c|c|c|}
                \hline

                $U,В$ & $I,мА$ & $W,мВт$ & $T_{ярк}, K$ & $T, K$ & $\varepsilon_T$\\ \hline
                $1.823 \pm 0.001$ & $0.529 \pm 0.001$ & $0.964 \pm 0.002$ & $1202 \pm 1$ & $1240 \pm 10$ & $0.137 \pm 0.005$\\ \hline
                $2.196 \pm 0.001$ & $0.569 \pm 0.001$ & $1.250 \pm 0.002$ & $1315 \pm 1$ & $1360 \pm 10$ & $0.156 \pm 0.005$\\ \hline
                $2.550 \pm 0.001$ & $0.604 \pm 0.001$ & $1.540 \pm 0.003$ & $1415 \pm 1$ & $1470 \pm 10$ & $0.172 \pm 0.005$\\ \hline
                $3.015 \pm 0.001$ & $0.649 \pm 0.001$ & $1.957 \pm 0.003$ & $1532 \pm 1$ & $1590 \pm 10$ & $0.191 \pm 0.005$\\ \hline
                $3.515 \pm 0.001$ & $0.694 \pm 0.001$ & $2.439 \pm 0.004$ & $1579 \pm 1$ & $1640 \pm 10$ & $0.199 \pm 0.005$\\ \hline
                $4.375 \pm 0.001$ & $0.768 \pm 0.001$ & $3.360 \pm 0.004$ & $1710 \pm 1$ & $1780 \pm 10$ & $0.220 \pm 0.006$\\ \hline
                $4.905 \pm 0.001$ & $0.811 \pm 0.001$ & $3.978 \pm 0.005$ & $1773 \pm 1$ & $1850 \pm 10$ & $0.230 \pm 0.006$\\ \hline
                $6.250 \pm 0.001$ & $0.913 \pm 0.001$ & $5.706 \pm 0.006$ & $1907 \pm 1$ & $1990 \pm 10$ & $0.252 \pm 0.006$\\ \hline
                $6.709 \pm 0.001$ & $0.945 \pm 0.001$ & $6.340 \pm 0.007$ & $1980 \pm 1$ & $2070 \pm 10$ & $0.264 \pm 0.006$\\ \hline
                $7.433 \pm 0.001$ & $0.995 \pm 0.001$ & $7.396 \pm 0.007$ & $2123 \pm 1$ & $2220 \pm 20$ & $0.287 \pm 0.007$\\ \hline
                $8.750 \pm 0.001$ & $1.080 \pm 0.001$ & $9.450 \pm 0.009$ & $2208 \pm 1$ & $2310 \pm 20$ & $0.301 \pm 0.007$\\ \hline

            \end{tabular}
            \caption{Результаты измерения для вольфрамовой нити}
            \label{tab:wolfram_res}
        \end{table}

\end{document}

